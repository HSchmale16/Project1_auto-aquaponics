\documentclass[american,man,12pt]{apa6}

\usepackage{hyperref}
\usepackage{babel}
\usepackage{csquotes}
\usepackage[backend=biber,date=short,maxcitenames=2,style=apa]{biblatex}

\DeclareLanguageMapping{american}{american-apa}

\title{Automated Aquaponics System Build and Setup Manual}
\shorttitle{AutoAquaponics}
\author{Henry J Schmale}
\affiliation{Harrisburg University of Science and Technology}

\addbibresource{project.bib}
\nocite{*}


\begin{document}
\maketitle
\tableofcontents

\section{Introduction}
This manual provides the documentation for my partially automated aquaponics
system. The system provides facilities for monitoring fish count, water level,
water temperature, and air quality factors. It also provides control systems
for toggling the pumps on and off, along with lighting factors for the tank.

\section{Software Architecture}



\section{Parts List}
\begin{itemize}
	\item Large Plastic Reservoir.
	\item 3 inch PVC Pipe, 5 foot long.
	\item Net Pots
	\item 390 Gallon Per Hour Submersible Pump
	\item 1/2 inch plastic tubing
    \item 1/2 inch barbed T-joint
    \item 2 - 1/2 inch hose barbs threaded
    \item Hose Clamps
\end{itemize}


\section{How To Build}
\begin{enumerate}
	\item Drill PVC pipe with hole saw. Start placing holes 7 inches from the
		  end, and then 6 inches on center. Continue until you have reached
		  the end of the pipe, but still have about 7 inches.
\end{enumerate}


\section{Setup of the System on First Run}
To be written


\section{Starting a new garden}
This section describes how to start a new crop in the automated aquaponics
system. It is broken down into two sections, the fish and starting the plants.

\begin{enumerate}
	\item Fill tank with water.
\end{enumerate}



\section{Troubleshooting}
To be written.


\printbibliography
\end{document}
